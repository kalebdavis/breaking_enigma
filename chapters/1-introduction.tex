\chapter{Introduction}\label{chap:introduction}

The Enigma cryptography machine was invented in 1918 by Arthur Sherbius. It played an integral role during World War II as Nazi Germany used it to encode its messages. The Germans were able to communicate for nearly 10 years, between the early 1920s and 1933, before Enigma was first broken by the Polish Cipher Bureau. The breaking of Enigma was not a singular event, but rather several major events that led up to the creation of automated machines that could universally decode Enigma messages. The breaking of Enigma stands to be an elemental achievement during the second World War. The intelligence gathered thanks to the team at Bletchley Park saved countless lives and gave Britain and its allies a distinct edge in the war. From a mathematical standpoint, the breaking of Enigma is significant for the the methods in which messages were decoded. From manual methods to increasingly more automated ones, the Polish Cipher Bureau and Bletchley Park both created some of the first automated decryption tools to aide the team in efficiently finding daily keys to ciphertext. Some of the largest breakthroughs in being able to break Enigma came from human error on the part of cipher clerks who would have habitual tendencies when enciphering messages, making it easier for the decryption teams to find keys. Several other fallacies in judgement from Germany’s military helped the teams become even more learned about Enigma’s inner workings. The combination of human error and cryptanalysis allowed the Polish and British teams to successfully break Enigma and automate their practices for finding message keys.
