\chapter{Examples}\label{chap:examples}

\section{Encryption Process}

In accordance with Kerchoff's principle, the method of encrypting a message was known by the Polish and British cryptologists attempting to break the cipher. In this paper, we will touch on two (2) separate methods of encrypting a message that were used during World War II. The first method was common practice until September 1938, when Germany decided to increase security by changing to the second method, which was used from then on.

\section{Encryption Process Pre-1938}\label{sec:encprocess1938}

Prior to the security changes in 1938, the process for encryption stayed mostly static. The encipherer, person who wanted to encrypt a message, would begin by setting their machine to the daily settings found in the widely dispersed codebook. These daily settings would correspond to initial settings of the rotors (which alphabetic character was visible from the top of the machine), as well as the settings of the plugboard (which letters would be transposed with which other letters). From there, the encipherer would choose their own individual key, which was supposed to be a random set of three (3) characters that was unique to the individual message. They would type their individual key into the Enigma machine twice, thus encrypting it twice using the daily settings. They typed it twice so as to ensure that there were no errors in encrypting the key, similar to how websites ask for a password confirmation on creation of a password. After encrypting the password using the daily settings, the encipherer would set the machine to their individual key, and encrypt the actual message. \cite{wt06}.

In order to decipher the message received, the decipherer would follow an almost identical process. They would set their machine to the daily key, and type the first six (6) letters of the message into the machine. This would reveal the individual key. After ensuring that it matched, the decipherer would set their machine to the individual key, and type in the rest of the message to reveal the plain text. Note that the same machine (the same internal hardware) is used to both encrypt and decrypt the message.

\section{Mathematical Theory Behind Cracking Enigma}

The initial goal in cracking the Enigma machine was to determine the hardware connections on the rotor closest to the keyboard. Knowing the hardware connections on that rotor would allow the Polish mathematicians to reconstruct their own Enigma machine, and use that to decrypt more messages. It would open the doors for more rapid and widespread decryption.

Of course, it was not easy to find out what the hardware connections on the rotor were. All the mathematicians had to work with were ciphertexts that had been intercepted throughout the day, so they could only attempt ciphertext-only attacks. However, Polish mathematicians Marian Rejewski, Jerzy Rozycki, and Henryk Zygalski were able to crack the Enigma cipher in 1932. The solution to this believed 'impossible' problem took its roots in the permutation theory that was covered in \secref{sec:mathconcepts}, with some help from the enciphering process itself, covered in \secref{sec:encprocess1938}.

Since every message was encrypted with a different individual key, coupled with the fact that each rotor rotated at different speeds, it was impossible to determine the plain text just by comparing ciphertexts. However, with knowledge of the encryption process, it was possible to draw conclusions about the first six (6) letters of each ciphertext, and with enough ciphertexts, draw conclusions about the inner workings of the Enigma machine. The mathematicians knew that the first six (6) letters were a repeat of the same three (3) letter key, they could draw two conclusions.

\begin{enumerate}
\item All message keys started from the same position, set from a codebook.
\item The first letter in the plaintext was the same as the fourth, the second the same as the fifth, and the third the same as the sixth.
\end{enumerate}

This is where permutation theory comes in. In this section of the paper, we will make reference to permutations $A$ - $F$, which correspond in kind to a different substitution cipher created by Enigma. $A$ corresponds to the transposition of letters (of the form $(ab)(cd)(ef)...$ where each letter occurs once) that occurs on the very first keypress in an encryption, which correlates to the first value in the encipherer's individual key. $B$ then corresponds to the transposition that occurs on the second keypress, $C$ on the third, and so on. One will notice that $A$ and $D$ both correspond to the first value in the encipherer's individual key, $B$ and $E$ correspond to the second, and $C$ and $F$ correspond to the third. This was one of the first conclusions drawn by the Polish mathematicians as well.

The first step to cracking Enigma involved gathering as many ciphertexts as possible and recognizing products of permutations within them. We know that $A$ and $D$ correspond to the same letter. This means that when the encipherer types a character $x$, he obtains the value $a$ as his ciphertext, and when he types the same character $x$ for the double enciphering of his individual key (in the fourth place), he obtains the value $b$, indicating a relationship between the values $a$ and $b$. This relationship between keys here can be modeled as a product of permutations $AD$, which is the product of the individual permutations $A$ and $D$.

As an example, take the ciphertexts

\begin{CVerbatim}[fontsize=\small]
dmq vbn
von puy
puc fmq
\end{CVerbatim}

One can see that $d \rightarrow v$, $v \rightarrow p$, and $p \rightarrow f$. That means that the permutation $AD$ contains $dvpf$. The same process can be applied to see that $oumb$ is in $BE$ and $cqny$ is in $CF$. If enough ciphertexts are gathered such that each letter of the alphabet is seen in each position at least once, entire permutations can be constructed. The permutation sets created from the daily ciphertexts are called the 'characteristic' for the day \cite{wk85}.

\begin{figure}[h!]
\begin{centering}
  \includegraphics[height=10cm]{images/permutations.jpg}
  \caption{Electric Current Through Enigma}
  \label{fig:current1}
\end{centering}
\end{figure}

Next, it is important to understand how the machine works, with regards to a circuit. \figref{fig:current1} shows the circuit that is created when a key is pressed. If we label the commutator (or plugboard) as $S$, the three rotors from left to right as $L$, $M$, $N$ respectively, and the reversing drum $R$, we can represent the path of the current as the product of the permutations $SNMLRL^{-1}M^{-1}N^{-1}S^{-1}$. However, we also need to account for the fact that the $N$ rotor revolving $1/26th$ of a turn each time a key is pressed, and we can represent that with the permutation $P = (abcdefghijklmnopqrstuvwxyz)$. Thus, if the $N$ rotor rotates twice, we will have $P^2$, and so on and so forth. With this, we can rewrite the previous equation for each individual keypress, including $P$.

$$A = SPNP^{-1}MLRL^{-1}M^{-1}PN^{-1}P^{-1}S^{-1}$$
$$B = SP^2NP^{-2}MLRL^{-1}M^{-1}P^2N^{-1}P^{-2}S^{-1}$$
$$C = SP^3NP^{-3}MLRL^{-1}M^{-1}P^3N^{-1}P^{-3}S^{-1}$$
$$D = SP^4NP^{-4}MLRL^{-1}M^{-1}P^4N^{-1}P^{-4}S^{-1}$$
$$E = SP^5NP^{-5}MLRL^{-1}M^{-1}P^5N^{-1}P^{-5}S^{-1}$$
$$F = SP^6NP^{-6}MLRL^{-1}M^{-1}P^6N^{-1}P^{-6}S^{-1}$$

It is clear that $MLRL^{-1}M^{-1}$ is repeated in every one of these equations, so we can replace that value with the value $Q$. \cite{wk85} We also must calculate the products $AD$, $BE$, and $CF$ with respect to the above equations, and we get the following equations:

$$AD = SPNP^{-1}QPN^{-1}P^3NP^{-4}QP^4N^{-1}P^{-4}S^{-1}$$
$$BE = SP^2NP^{-2}QP^2N^{-1}P^3NP^{-5}QP^5N^{-1}P^{-5}S^{-1}$$
$$CF = SP^3NP^{-3}QP^3N^{-1}P^3NP^{-6}QP^6N^{-1}P^{-6}S^{-1}$$

In order to solve these equations, we can take one of two routes. One, we can solve for $S$, $N$, and $Q$ using $AD$, $BE$, and $CF$. Two, we can solve for $A$ - $F$, $S$, $Q$, and $N$. One may notice that there are more unknowns than equations, which makes these sets impossible to solve. This is where the Polish mathematicians hit a stopping point. That is, until the French Cipher Bureau was able to provide the Polish Cipher Bureau with some codebooks that they had recovered from the Germans \cite{wk85}. This gave the Polish mathematicians the values of $S$ that they needed. They were also able to determine $A$ - $F$ based on encipherer's habits, and applications of the converse Theorem on the Product of Transpositions discussed in \secref{sec:mathconcepts}. Therefore, the four (4) unknowns from the equations was reduced to two (2), which is solvable.

$$SAS^{-1} = PNP^{-1}QPN^{-1}P^{-1}$$
$$SBS^{-1} = P^2NP^{-2}QP^2N^{-1}P^{-2}$$
$$SCS^{-1} = P^3NP^{-3}QP^3N^{-1}P^{-3}$$
$$SDS^{-1} = P^4NP^{-4}QP^4N^{-1}P^{-4}$$
$$SES^{-1} = P^5NP^{-5}QP^5N^{-1}P^{-5}$$
$$SFS^{-1} = P^6NP^{-6}QP^6N^{-1}P^{-6}$$

$N$ and $Q$ were able to be solved using the known values, and the resulting $N$ permutation corresponded to the connectors in that specific rotor \cite{wk85}. That $N$ was the result the mathematicians were after, and it was this permutation that was integral to continuous decryption of messages throughout the 1930s.

\section{Encryption Process Post-1938}

\section{Cryptological Machines}
