\chapter{Conclusion}\label{chap:conclusion}

When Germany caught wind of the fact that Enigma could be broken, they attempted different methods to increase security, such as changing the way the key was created, or changing the rotor arrangement \cite{wk85}. Each of these attempts at increasing security only set the Poles back a week or so at most, and soon they were back to being able to crack the codes just as they were before. In fact, when the Germans changed the rotor arrangement, it actually helped the Polish mathematicians. Moving the known $N$ rotor to another position meant that they already knew the connections of that rotor, and then they could use the same method as before to figure out the new rotor that was closest to the keyboard. This simple mistake by Germany, coupled with the habits of the encipherers when creating their individual keys, brings to the forefront an idea that is important to grasp from this paper. Enigma would have been unbreakable if it had not been for the encipherers' habits and predictability. Human error was the only real issue with the cryptosystem. This is the case for many cryptosystems: they are theoretically incredibly secure, but in practice are misused. It would be easy to blame Germany's lack of training the encipherers on the reason there were so many habitual individual keys, but that would be naive. Humans are as a rule habitual beings, and so it would be interesting to try to create a cryptosystem that completely erases human vulnerability.

To conclude, Enigma was an incredible asset to the Germans during World War Two, and was an even greater asset to the Allies when they were able to crack it. It was also a large step in cryptological machines, as it was one of the first widely believed 'unbreakable' cipher machines in existence. The machine spurred new technological advancements such as the cryptological bombes, including the one created by Alan Turing. Without this cryptological machine, we might not be where we are today with regards to technology, so we have that to thank for it.
